%%
%% This is file `thesis.tex',
%% generated with the docstrip utility.
%%
%% The original source files were:
%%
%% nudtpaper.dtx  (with options: `thesis')
%% 
%% This is a generated file.
%% 
%% Copyright (C) 2018 by TomHeaven <hanlin_tan@nudt.edu.cn>
%% 
%% This file may be distributed and/or modified under the
%% conditions of the LaTeX Project Public License, either version 1.3a
%% of this license or (at your option) any later version.
%% The latest version of this license is in:
%% 
%% http://www.latex-project.org/lppl.txt
%% 
%% and version 1.3a or later is part of all distributions of LaTeX
%% version 2004/10/01 or later.
%% 
%% To produce the documentation run the original source files ending with `.dtx'
%% through LaTeX.
%% 
%% Thanks LiuBenYuan <liubenyuan@gmail.com> for maintainence.
%% Thanks Xue Ruini <xueruini@gmail.com> for the thuthesis class!
%% Thanks sofoot for the original NUDT paper class!
%% 
%1. 规范硕士导言
% \documentclass[master,ttf]{nudtpaper}
%2. 规范博士导言
% \documentclass[doctor,twoside,ttf]{nudtpaper}
%3. 如果使用是Vista
% \documentclass[master,ttf,vista]{nudtpaper}
%4. 建议使用OTF字体获得较好的页面显示效果
%   OTF字体从网上获得,各个系统名称统一,不用加vista选项
%   如果你下载的是最新的(1201)OTF英文字体,建议修改nudtpaper.cls,使用
%   Times New Roman PS Std
% \documentclass[doctor,twoside,otf]{nudtpaper}
%   另外,新版的论文模板提供了方正字体选项FZ,效果也不错哦
% \documentclass[doctor,twoside,fz]{nudtpaper}
%5. 如果想生成盲评,传递anon即可,仍需修改个人成果部分
% \documentclass[master,otf,anon]{nudtpaper}
%
\documentclass[doctor,ttf]{nudtpaper}
%\usepackage{lmodern} % 这个包导致英文字体与Word的渲染效果不一致,故去掉
\usepackage{mynudt}
\usepackage{multirow,array,makecell}
\usepackage{fix-cm}
\usepackage{hyperref}
%\usepackage{autobreak}
\usepackage{stmaryrd}
\usepackage[ruled,linesnumbered,vlined,algochapter]{algorithm2e}
\usepackage{tikz}
\usetikzlibrary{automata, positioning, fit, shapes}
\newcommand{\Bullet}{{\fontsize{6pt}{6pt}\selectfont\CircleSolid}}
\newcommand\cox{%
        \mathrel{\text{\raisebox{.7ex}{\tikz[baseline] \draw (0,0) circle (1ex) (0ex,1ex) -- (-1ex,0ex) (-1ex,0ex) -- (0ex,-1ex) (0ex,-1ex) -- (1ex,0ex) (1ex,0ex) -- (0ex,1ex);}%
}}}
\newcommand\coxe{%
        \mathrel{\text{\raisebox{.7ex}{\tikz[baseline] \draw (0,0) circle (1ex) (0.707ex,0.707ex) -- (-0.707ex,0.707ex) (-0.707ex,0.707ex) -- (-0.707ex,-0.707ex) (-0.707ex,-0.707ex) -- (0.707ex,-0.707ex) (0.707ex,-0.707ex) -- (0.707ex,0.707ex);}%
}}}
\newcommand\Future{%
        \mathrel{\text{\raisebox{.7ex}{\tikz[baseline] \draw (0ex,1ex) -- (-1ex,0ex) (-1ex,0ex) -- (0ex,-1ex) (0ex,-1ex) -- (1ex,0ex) (1ex,0ex) -- (0ex,1ex);}%
}}}
\newcommand\Global{%
        \mathrel{\text{\raisebox{.7ex}{\tikz[baseline] \draw (0.707ex,0.707ex) -- (-0.707ex,0.707ex) (-0.707ex,0.707ex) -- (-0.707ex,-0.707ex) (-0.707ex,-0.707ex) -- (0.707ex,-0.707ex) (0.707ex,-0.707ex) -- (0.707ex,0.707ex);}%
}}}
\newcommand\Next{%
        \mathrel{\text{\raisebox{.7ex}{\tikz[baseline] \draw (0,0) circle (1ex);}%
}}}
\newcommand\Previous{%
        \mathrel{\text{\raisebox{.7ex}{\tikz[baseline] \draw (0,0) circle (1ex) (-0.65ex,0ex) -- (0.65ex,0ex);}%
}}}
\newcommand\Once{%
        \mathrel{\text{\raisebox{.7ex}{\tikz[baseline] \draw (0ex,1ex) -- (-1ex,0ex) (-1ex,0ex) -- (0ex,-1ex) (0ex,-1ex) -- (1ex,0ex) (1ex,0ex) -- (0ex,1ex) (-0.65ex,0ex) -- (0.65ex,0ex);}%
}}}
\newcommand\Historical{%
        \mathrel{\text{\raisebox{.7ex}{\tikz[baseline] \draw (0.707ex,0.707ex) -- (-0.707ex,0.707ex) (-0.707ex,0.707ex) -- (-0.707ex,-0.707ex) (-0.707ex,-0.707ex) -- (0.707ex,-0.707ex) (0.707ex,-0.707ex) -- (0.707ex,0.707ex) (-0.65ex,0ex) -- (0.65ex,0ex);}%
}}}

\classification{TP301.2}
\serialno{17069032}
\confidentiality{公开}
\UDC{004.41}
\title{面向不可综合规约的无人系统控制程序\\自动生成方法研究}
\displaytitle{面向不可综合规约的无人系统控制程序自动生成方法研究}
\author{史浩}
\zhdate{\zhtoday}
\entitle{Automatic Generation of Control Program for Unmanned System with Unsynthesizable Specifications}
\enauthor{Hao Shi}
\endate{\entoday}
\subject{软件工程}
\ensubject{Software Engineering}
\researchfield{高可信软件技术}
\supervisor{董威\quad{}教授}
\cosupervisor{}  % 协助指导教师,没有就空着
\ensupervisor{Prof. Wei Dong}
\encosupervisor{} % 协助指导教师英文,没有就空着
\papertype{工学}
\enpapertype{Engineering}
% 加入makenomenclature命令可用nomencl制作符号列表。

\begin{document}
	\graphicspath{{figures/}}
	% 制作封面,生成目录,插入摘要,插入符号列表 \\
	% 默认符号列表使用denotation.tex,如果要使用nomencl \\
	% 需要注释掉denotation,并取消下面两个命令的注释。 \\
	% cleardoublepage% \\
	% printnomenclature% \\
	\maketitle
	\frontmatter
	\tableofcontents
	\listoftables
	\listoffigures
	
	\midmatter
	\begin{cabstract}
    中文摘要
\end{cabstract}
\ckeywords{关键词1;关键词2;关键词3}

\begin{eabstract}
    english abstarct
\end{eabstract}
\ekeywords{Keyword1, Keyword2, Keyword3}


	\input{data/denotation}
	
	%书写正文,可以根据需要增添章节; 正文还包括致谢,参考文献与成果
	\mainmatter
	\renewcommand\UrlFont{\timesnr}
	\makeatletter
	\newcounter{blankpages}
	\def\cleardoublepage{%
		\clearpage
		\if@twoside
		\ifodd\c@page
		\else
		\hbox{}\newpage\stepcounter{blankpages}%
		\thispagestyle{empty}%
		\if@twocolumn\hbox{}\newpage\fi
		\fi
		\fi
	}
	\newcommand{\@romannoblank}[1]{%
		\@roman{\numexpr#1-\value{blankpages}\relax}%
	}
	\makeatother
	
	\input{data/chap01}
	\input{data/chap02}
	\input{data/chap03}
	\input{data/chap04}
	\input{data/chap05}
	\input{data/chap06}
	
%%% Local Variables:
%%% mode: latex
%%% TeX-master: "../main"
%%% End:

\begin{ack}
    % 听我说,谢谢你,因为有你,温暖了四季
    % 谢谢你,因为有你 ~~
\end{ack}

	
	\cleardoublepage
	\phantomsection
	\addcontentsline{toc}{chapter}{参考文献}
	\bibliographystyle{bstutf8}
	\bibliography{ref/refs}
	
	\input{data/resume_blind}
	% 最后,需要的话还要生成附录,全文随之结束。
	\appendix
	\backmatter
	% TeX
\chapter{关键算法的代码实现}

\section{算法的Python代码实现}

\begin{lstlisting}[language={Python}]
    def Generate_SCC(r):
        global G, index, S, DFN, Low, Color
        DFN[r] = index
        Low[r] = index
        index += 1
        S.append(r)
        Color[r] = 'grey'
        for tran in Depmap[r]:
            rj = tran[1]
            if Color[rj] == 'white':
                Generate_SCC(rj)
                Low[r] = min(Low[r], Low[rj])
            elif Color[rj] == 'grey':
                Low[r] = min(Low[r], DFN[rj])
        if Low[r] == DFN[r]:
            Vi = []
            while True:
                rs = S.pop()
                Color[rs] = 'black'
                Vi.append(rs)
                if rs == r:
                    break
            Gi = dict.fromkeys(Vi)
            for rx in Vi:
                Gi[rx] = []
            for rx in Vi:
                for tranx in Depmap[rx]:
                    if tranx[1] in Vi:
                        Gi[rx].append(tranx)
            G.append(Gi)   
\end{lstlisting}

	
\end{document}
